\documentclass[../main.tex]{subfiles}
\begin{document}
In order to make sure we are correctly benchmarking the performance impact of CCAs rather than miscellneous confounding factors, we ran the benchmark on an out-of-the-box Debian image on a VM. To ensure the CPU benchmark and the CCA are running on the same CPU, our VM is set up to contain only one core. We also decided to run our benchmarks on various implementations of TCP Cubic due to its wide availability across platforms. \\
The results of behavioral benchmarks are shown in Figure \ref{fig:behavioral}. From the graph, we can see that [to-be-added-conclusion]. The results of CPU benchmarks are shown in Figure \ref{fig:cpu}. The eBPF implementation of TCP Cubic achieved impressive CPU performance compared to the built-in Linux kernel networking stack implementation. CCP-Cubic from the CCP Project (cite generic-cong-avoid git repo) combined with the kernel module + netlink CCP datapath achieved slightly lower performance, where its median is at x percent of the full kernel implementation value. Our implementation of eBPFCCP ranked lower on the performance chart, but was still able to achieve y percent of the full kernel implementation value. Due to the relative comparability of CPU performance between the kernel and eBPF implementations of TCP Cubicm, we theorize the noticable gap between eBPFCCP and kernel module CCP was due to our choice of using unix pipes rather than netlink as a channel for communicating between the eBPFCCP and the user space program. \\
From these results, we can conclude that eBPF is a viable platform for the development CCAs. The negligible performance overhead when compared to the kernel native implementation is impressive, and suggests that eBPF-CCAs can possibly also be used as a long-term deployment option outside of research and development stages. 
\end{document}
