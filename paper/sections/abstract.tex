\documentclass[../main.tex]{subfiles}
\begin{document}
Implementing new CCAs (congestion control algorithms) is traditionally a time-consuming and error-prone process. Developing new CCAs directly in the kernel not only requires the developer to have advanced knowledge about kernel programming, but also still introduces opportunities for potentially fatal bugs that will crash the kernel. Furthermore, the development and testing process is made slow from the constant need to reboot and difficulties in running. While developing CCAs as kernel modules (https://pbg.cs.illinois.edu/papers/jay18pcc-kernel.pdf) has traditionally been an accepted alternative, running developmental software directly in the kernel still introduces instability and crashes. Recent research, such as CCP (CCP), explored the possibility of developing and running CCAs in userspace, having the algorithm communicate with the kernel through a developer-friendly datapath api that is vetted for safety. Simultaneously, developing CCAs in eBPF (https://dl.acm.org/doi/10.1145/3609021.3609295) has been explored as a new alternative as the eBPF verifier checks the provided program for issues like memory-safety violations and infinite loops. In this paper, we will evaluate the different approaches to CCA development by measuring their behavioral correctness and performance cost. 
\end{document}
