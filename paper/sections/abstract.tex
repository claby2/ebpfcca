\documentclass[../main.tex]{subfiles}
\begin{document}
Developing new Congestion Control Algorithms (CCAs) has traditionally been a time-consuming and error-prone process. Creating these algorithms directly within the Linux kernel demands a high level of kernel programming expertise while still leaving room for bugs that can crash the system. Although implementing CCAs as kernel modules has traditionally been an accepted alternative (https://pbg.cs.illinois.edu/papers/jay18pcc-kernel.pdf), it still runs the risk of introducing instability and crashes. Recent research has implemented a Congestion Control Plane (CCP) (https://akshayn.xyz/res/ccp-sigcomm18.pdf) that allows CCA developers to implement CCAs in userspace, eliminating the need to run experimental software directly in the kernel. This approach involves a carefully designed and safety-checked datapath API; however, the Linux kernel datapath API still relies on a kernel module. Simultaneously, developing CCAs in eBPF (https://dl.acm.org/doi/10.1145/3609021.3609295) has emerged as another promising approach, leveraging the eBPF verifier to ensure safety before the code ever runs in the kernel. In this paper, we compare these various approaches -- kernel-based, eBPF-based, and userspace (via CCP) -- along with a new hybrid userspace eBPF-based approach (ebpfccp) for behavioral correctness and performance overhead.
\end{document}
