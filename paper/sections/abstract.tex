\documentclass[../main.tex]{subfiles}
\begin{document}
This paper seeks to explore the viability of developing Congestion Control Algorithms (CCAs) via eBPF. Developing new CCAs has traditionally been a time-consuming and error-prone process. Creating these algorithms directly within the Linux kernel demands a high level of kernel programming expertise while still leaving room for bugs that can crash the system. Although implementing CCAs as kernel modules has traditionally been an accepted alternative \cite{Jay2018APK}, it still runs the risk of introducing instability and crashes. Recent research has implemented a Congestion Control Plane (CCP) \cite{ccp} that allows CCA developers to implement CCAs in user space, eliminating the need to run experimental software directly in the kernel. This approach involves a carefully designed and safety-checked datapath API; however, the Linux kernel datapath API still relies on a kernel module. Simultaneously, developing CCAs in eBPF \cite{ebpf_cca} has emerged as another promising approach, leveraging the eBPF verifier to ensure safety before the code ever runs in the kernel. In this paper, we compare these various approaches -- kernel-based, eBPF-based, and user space (via CCP) -- along with a new hybrid user space eBPF-based approach (ebpfccp) in terms of behavioral correctness and performance overhead.
\end{document}
