\documentclass[../main.tex]{subfiles}
\begin{document}
Congestion Control Algorithms form a critical part of the Internet ecosystem. While traditional discussions about CCAs centered around ideas like fairness between users (cite flow-based fairness/tcp friendly paper here), newer research explored potentials for CCAs to improve network performance (cite BBR?) and allow the user to customize bandwidth allocated to different applications on the same machine. Research on various old and new congestion control algorithms is critical to the stable operation and continuous improvement of the Internet. \\
The development of new CCAs faced two problems. The first problem being the high friction of development: CCAs have traditionally ran inside the networking stack in the kernel. As a result, developer of new CCAs are required to program in a kernel environment, forming a high barrier of entry and enabling potential high-impact bugs. The second problem being the difficulties of distribution. Though some have argued for abolishing the idea of flow-based fairness (Dismantling), TCP-Friendliness has nevertheless remained a central piece for any CCA-related research. To correctly evaluated the impacts of a new CCA on the Internet ecosystem, we need to be able to see significant real-world adoption from actual users around the world. Pushing a new, untested CCA upstream to the Linux kernel is an extremely difficult, if not outright impossible task. As a result, any CCA wishing to see eventual large-scale adoption must first propose a relatively user-friendly way for people to adopt it in the early stages. \\
We believe that eBPF as a platform for implementing congestion control solves both problems. On the friction side, while programming in eBPF still requires a non-negligible amount of domain expertise, the process is nevertheless much simpler when compared to developing kernel modules, for example. The runtime-attached nature of eBPF makes iterating and testing new versions much easier. Furthermore, the eBPF VM and Verifier provide powerful guarantees to the safety of the program, protecting against large amounts of hard-to-solve bugs that lead to the kernel crashing. On the adoption side, the safety guarantees provided by eBPF will greatly encourage user adoption. While users can be hesitant in installing kernel patches or modules, attaching a eBPF program is much lower-risk and appealing if the new CCA itself promises significant improvements.\\
Despite these upsides, running CCAs in eBPF can possibly incur extra costs due to running in the eBPF VM. By instrumenting the performance overheads of various different ways of implementing CCAs, we would like to explore whether eBPF is a viable long-term platform to implement CCAs on, or would researchers be better-off to switch to a kernel space implementation once a proof-of-concept was done on eBPF. In addition to CCAs directly implemented in eBPF, we also tested other means of developing CCAs quickly and safely. The Congestion Control Plane (CCP) (cite CCP), for example, is an idea proposed to implement CCAs in user space. Since the original CCP kernel datapath is written as a Linux kernel module, we also implemented an alternative datapath in eBPF to potentially take advantage of the safety guarantees.
\end{document}
